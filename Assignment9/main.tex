\documentclass[journal,12pt,twocolumn]{IEEEtran}
%

\usepackage{setspace}
\usepackage{gensymb}
\singlespacing

\usepackage{amsmath}
\usepackage{amsthm}
\usepackage{txfonts}
\usepackage{cite}
\usepackage{enumitem}
\usepackage{mathtools}
\usepackage{listings}
    \usepackage{color}                                            %%
    \usepackage{array}                                            %%
    \usepackage{longtable}                                        %%
    \usepackage{calc}                                             %%
    \usepackage{multirow}                                         %%
    \usepackage{hhline}                                           %%
    \usepackage{ifthen}                                           %%
  %optionally (for landscape tables embedded in another document): %%
    \usepackage{lscape}     
\usepackage{multicol}
\usepackage{chngcntr}
\usepackage{float}
\renewcommand\thesection{\arabic{section}}
\renewcommand\thesubsection{\thesection.\arabic{subsection}}
\renewcommand\thesubsubsection{\thesubsection.\arabic{subsubsection}}

\renewcommand\thesectiondis{\arabic{section}}
\renewcommand\thesubsectiondis{\thesectiondis.\arabic{subsection}}
\renewcommand\thesubsubsectiondis{\thesubsectiondis.\arabic{subsubsection}}

% correct bad hyphenation here
\hyphenation{op-tical net-works semi-conduc-tor}
\def\inputGnumericTable{}                                 %%

\lstset{
%language=C,
frame=single, 
breaklines=true,
columns=fullflexible
}

\begin{document}
%


\newtheorem{theorem}{Theorem}[section]
\newtheorem{problem}{Problem}
\newtheorem{proposition}{Proposition}[section]
\newtheorem{lemma}{Lemma}[section]
\newtheorem{corollary}[theorem]{Corollary}
\newtheorem{example}{Example}[section]
\newtheorem{definition}[problem]{Definition}
\newcommand{\BEQA}{\begin{eqnarray}}
\newcommand{\EEQA}{\end{eqnarray}}
\newcommand{\define}{\stackrel{\triangle}{=}}
\bibliographystyle{IEEEtran}
\providecommand{\mbf}{\mathbf}
\providecommand{\pr}[1]{\ensuremath{\Pr\left(#1\right)}}
\providecommand{\qfunc}[1]{\ensuremath{Q\left(#1\right)}}
\providecommand{\sbrak}[1]{\ensuremath{{}\left[#1\right]}}
\providecommand{\lsbrak}[1]{\ensuremath{{}\left[#1\right.}}
\providecommand{\rsbrak}[1]{\ensuremath{{}\left.#1\right]}}
\providecommand{\brak}[1]{\ensuremath{\left(#1\right)}}
\providecommand{\lbrak}[1]{\ensuremath{\left(#1\right.}}
\providecommand{\rbrak}[1]{\ensuremath{\left.#1\right)}}
\providecommand{\cbrak}[1]{\ensuremath{\left\{#1\right\}}}
\providecommand{\lcbrak}[1]{\ensuremath{\left\{#1\right.}}
\providecommand{\rcbrak}[1]{\ensuremath{\left.#1\right\}}}
\theoremstyle{remark}
\newtheorem{rem}{Remark}
\newcommand{\sgn}{\mathop{\mathrm{sgn}}}
\providecommand{\abs}[1]{\left\vert#1\right\vert}
\providecommand{\res}[1]{\Res\displaylimits_{#1}} 
\providecommand{\norm}[1]{\left\lVert#1\right\rVert}
\providecommand{\mtx}[1]{\mathbf{#1}}
\providecommand{\mean}[1]{E\left[ #1 \right]}
\providecommand{\fourier}{\overset{\mathcal{F}}{ \rightleftharpoons}}
\providecommand{\system}{\overset{\mathcal{H}}{ \longleftrightarrow}}


\newcommand{\myvec}[1]{\ensuremath{\begin{pmatrix}#1\end{pmatrix}}}
\newcommand{\cmyvec}[1]{\ensuremath{\begin{pmatrix*}[c]#1\end{pmatrix*}}}
\newcommand{\mydet}[1]{\ensuremath{\begin{vmatrix}#1\end{vmatrix}}}
\newcommand{\proj}[2]{\textbf{proj}_{\vec{#1}}\vec{#2}}
\let\StandardTheFigure\thefigure
\let\vec\mathbf
 
\title{ASSIGNMENT 9}
\author{Gayathri S}
	

\maketitle
\renewcommand{\thefigure}{\theenumi}
\renewcommand{\thetable}{\theenumi}
  
   Download all python codes from 
\begin{lstlisting}
https://github.com/Gayathri1729/SRFP/tree/main/Assignment9
\end{lstlisting}
%
and latex-tikz codes from 
%
\begin{lstlisting}
https://github.com/Gayathri1729/SRFP/tree/main/Assignment9
\end{lstlisting}
%
\section{Matrices 2.67}
Express the matrix $\vec{B}$ = $
\begin{pmatrix} 
2 & -2 & -4 \\  -1 & 3 & 4 \\ 1&-2&-3
\end{pmatrix}$
 as the sum of a symmetric and a skew symmetric matrix.
\section{Solution}
Given $\vec{B}$ = $
\begin{pmatrix} 
2 & -2 & -4 \\  -1 & 3 & 4 \\ 1&-2&-3
\end{pmatrix}$

Let $\vec{C}$ = $\frac{\vec{B}+\vec{B^\top}}{2}$ and $\vec{D}$ = $\frac{\vec{B}-\vec{B^\top}}{2}$.


We know that $\frac{\vec{B}+\vec{B^\top}}{2}$ is a symmetric matrix and $\frac{\vec{B}-\vec{B^\top}}{2}$ is a skew symmetric matrix.

\begin{equation}
\vec{B^\top} = 
\begin{pmatrix} 
2 & -1 & 1 \\ -2 & 3 & -2 \\ -4 & 4 & -3
\end{pmatrix} 
\end{equation} 

\begin{align}
  \vec{C} &= \frac{\vec{B}+\vec{B^\top}}{2} = \begin{pmatrix} 
2 & -\frac{3}{2} & -\frac{3}{2} \\  -\frac{3}{2} & 3 & 1 \\ -\frac{3}{2}&1&-3
\end{pmatrix} \label{eq1}
\end{align}


\begin{align}
\vec{D} &=\frac{\vec{B} - \vec{B^\top}}{2} =
\begin{pmatrix} 
0 & -\frac{1}{2} & -\frac{5}{2} \\  \frac{1}{2} & 0 & 3 \\ \frac{5}{2}&-3&0
\end{pmatrix} \label{eq2}
\end{align}
Note that $\vec{C} + \vec{D}$ = $\vec{B}$.

Hence from \eqref{eq1} and \eqref{eq2}, $\vec{B}$ can be expressed as the sum of a symmetric and a skew symmetric matrix.
\end{document}
